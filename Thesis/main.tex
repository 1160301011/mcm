\documentclass{mcmthesis}
\mcmsetup{CTeX = false,   % 使用 CTeX 套装时,设置�? true
        tcn = 1912531, problem = B,% 队伍控制号码,接受一个字符串作为值;选题,接受一个字符串作为值;
        sheet = true, %为真时将输出摘要页,否则不输出;默认�? true�?
        color = red,  %设置控制页的题目号的颜色
        titleinsheet = true, %为真时将在摘要页输出标题,否则不输出;默认为 false�?
        keywordsinsheet = true,%为真时将在摘要页输出关键字,否则不输出;默认�? false�?
        titlepage = true,%为真时将输出标题页,否则不输出;默认�? true�?
        abstract = true}%为真时将在标题页输出摘要和关键词,否则不输出;默认值为 true�?
\usepackage[numbers,sort&compress]{natbib}
\usepackage{times}
\usepackage{mathptmx}
\usepackage{geometry}
\usepackage{cite}
\usepackage{graphicx}
\usepackage{subfigure}
\usepackage{indentfirst}
\usepackage{chngpage}
\usepackage{array}
\usepackage{booktabs}
\usepackage{threeparttable}
\usepackage{longtable}
\usepackage{amsmath}
\title{\LARGE Modeling on Drone-based Disaster Response System}
\author{ }
\date{\today}
\makeatletter

\usepackage{titletoc}
\titlecontents{section}[2cm]{\bf}{\contentslabel{2em}}{}{%
\titlerule*[0.5pc]{$\cdot$}\contentspage}%
\titlecontents{subsection}[3cm]{\normalsize}{\contentslabel{2em}}{}{%
\titlerule*[0.5pc]{$\cdot$}\contentspage}%
\titlecontents{subsubsection}[5.3cm]{\normalsize}{\contentslabel{3.0em}}{}{%
\titlerule*[0.5pc]{$\cdot$}\contentspage}%

\newtheorem{assumption}{Assumption}%假设编号
\newtheorem{theorem}{Theorem}%定理编号
\newtheorem{lemma}{Lemma}%推论编号

\begin{document}
    \begin{abstract}

        
    \end{abstract}

    \begin{keywords}
        
    \end{keywords}

    \maketitle
    \tableofcontents
    \section{Introdution}
    \subsection{Background}

    \subsection{Problem Restatment}

    \subsection{Our Work}

    \section{Assumptions}
    \begin{assumption}
        %Assumptions
    \end{assumption}

    \section{List of Notations}
    \begin{center}
        \begin{longtable}{cl}
            \caption{The List of Notation}
            \hline
            Symbol & Meaning
            \hline

        \end{longtable}

    \end{center}

    \begin{table}[!ht]
        \caption{The categories of hurricanes}
         \renewcommand\arraystretch{1.5}
         \setlength{\abovecaptionskip}{0pt}%    
        \setlength{\belowcaptionskip}{10pt}%
        \begin{center}
        \begin{tabular}{p{.2\textwidth}p{.2\textwidth}m{.5\textwidth}}
        \toprule[1.5pt]
        Container & 1 & 2 & 3 \\
         \midrule
        
          A & 0 & 0 & 0  \\
        
          B & 1 & 3 & 1  \\
        
          C & 0 & 2 & 0  \\
        
          D & 0 & 0 & 0 \\
        
          E & 0 &  0 & 0\\
          
          F & 0 &  0 & 1 \\
        
          G & 0 &  0 & 0 \\
        
          H & 3 & & 5 & 1\\
        
          MED1 &1925 &252 &637 \\
        
          MED2 &0 & 690&0 \\
        
          MED3 &0 &536 &1903 \\
        
          Fill Rate & 91.79\% & 98.7\% & 71.2\% \\
        
          Days it can serve & 2839 & 50 & 449\\
        
          \bottomrule[1.5pt]
         \end{tabular}
         \end{center} 
         \end{table}

         \begin{table}[!ht]
            \caption{The categories of hurricanes}
             \renewcommand\arraystretch{1.5}
             \setlength{\abovecaptionskip}{0pt}%    
            \setlength{\belowcaptionskip}{10pt}%
            \begin{center}
            \begin{tabular}{p{.2\textwidth}p{.2\textwidth}m{.5\textwidth}}
            \toprule[1.5pt]
             Container & A & B & C & D & E & F & G & H\\
             \midrule
            
              1 & 0 & 1 & 0 & 0 & 0 & 0 & 0 & 3  \\
            
              2 & 0 & 3 & 2 & 0 & 0 & 0 & 0 & 5  \\
            
              3 & 0 & 0 & 0 & 0 & 0 & 1 & 0 & 1  \\
            
              \bottomrule[1.5pt]
             \end{tabular}
             \end{center} 
             \end{table}

             \begin{table}[!ht]
                \caption{The categories of hurricanes}
                 \renewcommand\arraystretch{1.5}
                 \setlength{\abovecaptionskip}{0pt}%    
                \setlength{\belowcaptionskip}{10pt}%
                \begin{center}
                \begin{tabular}{p{.2\textwidth}p{.2\textwidth}m{.5\textwidth}}
                \toprule[1.5pt]
                 Container & MED1 & MED2 & MED3\\
                 \midrule
                
                  1 & 1 & 0 & 0   \\
                
                  2 & 5 & 3 & 2   \\
                
                  3 & 1 & 0 & 1   \\
                
                  \bottomrule[1.5pt]
                 \end{tabular}
                 \end{center} 
                 \end{table}

                 \begin{table}[!ht]
                    \caption{The categories of hurricanes}
                     \renewcommand\arraystretch{1.5}
                     \setlength{\abovecaptionskip}{0pt}%    
                    \setlength{\belowcaptionskip}{10pt}%
                    \begin{center}
                    \begin{tabular}{p{.2\textwidth}p{.2\textwidth}m{.5\textwidth}}
                    \toprule[1.5pt]
                     Container & A & B & C & D & E & F & G & H\\
                     \midrule
                    
                      1 & 0 & 1 & 0 & 0 & 0 & 0 & 0 & 3  \\
                    
                      2 & 0 & 3 & 2 & 0 & 0 & 0 & 0 & 5  \\
                    
                      3 & 0 & 1 & 0 & 0 & 0 & 1 & 0 & 1  \\
                    
                      \bottomrule[1.5pt]
                     \end{tabular}
                     \end{center} 
                     \end{table}

                     \begin{table}[!ht]
                        \caption{The categories of hurricanes}
                         \renewcommand\arraystretch{1.5}
                         \setlength{\abovecaptionskip}{0pt}%    
                        \setlength{\belowcaptionskip}{10pt}%
                        \begin{center}
                        \begin{tabular}{p{.2\textwidth}p{.2\textwidth}m{.5\textwidth}}
                        \toprule[1.5pt]
                         State No.($b_i$) & 1 & 2 & 3 & 4 & 5 & 6 \\
                         \midrule
                        
                          Parallel to L & $l_{tk}$ & $w_{tk}$ & $h_{tk}$ & $l_{tk}$ & $w_{tk}$ & $h_{tk}$   \\
                        
                          Parallel to W & $w_{tk}$ & $l_{tk}$ & $l_{tk}$ & $h_{tk}$ & $h_{tk}$ & $w_{tk}$   \\
                        
                          Parallel to H & $h_{tk}$ & $h_{tk}$ & $w_{tk}$ & $w_{tk}$ & $l_{tk}$ & $l_{tk}$ \\
                        
                          \bottomrule[1.5pt]
                         \end{tabular}
                         \end{center} 
                         \end{table}

\end{document}